\documentclass[coursework, och]{SCWorks1}
\usepackage[T2A]{fontenc}
\usepackage[utf8]{inputenc}
\usepackage{graphicx}

\usepackage[sort,compress]{cite}
\usepackage{amsmath}
\usepackage{amssymb}
\usepackage{amsthm}
\usepackage{fancyvrb}
\usepackage{longtable}
\usepackage{array}
\usepackage[english,russian]{babel}

\usepackage[colorlinks=false]{hyperref}

\newcommand{\eqdef}{\stackrel {\rm def}{=}}

\newtheorem{lem}{Лемма}
\newtheorem{definition}{Определение}

\begin{document}


\begin{center}
    МИНОБРНАУКИ РОССИИ\\
    Федеральное государственное бюджетное образовательное учреждение высшего образования \\
    <<САРАТОВСКИЙ НАЦИОНАЛЬНЫЙ ИССЛЕДОВАТЕЛЬСКИЙ ГОСУДАРСТВЕННЫЙ УНИВЕРСИТЕТ ИМЕНИ Н.Г. ЧЕРНЫШЕВСКОГО>> \\
\end{center}
\begin{center}
Кафедра системного анализа и автоматического управления
\end{center}

\begin{center}
Отчет по заданию 5. Вариант 13
\end{center}

\begin{flushleft}
Студента 3 курса 321 группы направления 09.03.01 ИВТ\\
Факультета компьютерных наук и информационных технологий\\
Чесакова Максима Евгеньевича
\end{flushleft}
%=================================================================================
\textbf{Задача №1}
Имеется не более чем счётный набор положительных действительных чисел $0 < \alpha_i < 1$ таких, что $\displaystyle\sum_{i} \alpha_i = 1$, заданный диаграммой переходов автомата. Решить задачу аналитически: построить попарно непересекающиеся открытые множества $U_i$ в $\mathbb{Z}_2$ такие, что $\mu(U_i) = \alpha_i$ для всех $i$, где $\mu$ --- нормализованная мера Хаара.
\begin{figure}[!ht]
	\centering
	\includegraphics[width=10cm]{Диаграмма.png}
	\caption{Проверка $h(x)$}
\end{figure}

\textbf{Аналитическое решение.}



\textbf{Ответ:} .

\textbf{Программная реализация на Python.}

Исходный код представлен в файле 4.py. Результаты его работы представлены на рисунках 1.


\begin{Verbatim}[numbers=left]


\end{Verbatim}


%  \bibitem{Ione} GraphOnLine : [интерактивный онлайн-конструктор графов]. – URL: https://graphonline.ru/?graph=WnqNelghOjfEvOqsZZcst (дата обращения: 20.09.2025).
%  \label{сайт1}
%  \bibitem{Itwo} Источник 2
%\end{thebibliography}

%===============================================================================
\end{document}