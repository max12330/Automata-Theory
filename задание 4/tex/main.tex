\documentclass[coursework, och]{SCWorks1}
\usepackage[T2A]{fontenc}
\usepackage[utf8]{inputenc}
\usepackage{graphicx}

\usepackage[sort,compress]{cite}
\usepackage{amsmath}
\usepackage{amssymb}
\usepackage{amsthm}
\usepackage{fancyvrb}
\usepackage{longtable}
\usepackage{array}
\usepackage[english,russian]{babel}

\usepackage[colorlinks=false]{hyperref}

\newcommand{\eqdef}{\stackrel {\rm def}{=}}

\newtheorem{lem}{Лемма}
\newtheorem{definition}{Определение}

\begin{document}


\begin{center}
    МИНОБРНАУКИ РОССИИ\\
    Федеральное государственное бюджетное образовательное учреждение высшего образования \\
    <<САРАТОВСКИЙ НАЦИОНАЛЬНЫЙ ИССЛЕДОВАТЕЛЬСКИЙ ГОСУДАРСТВЕННЫЙ УНИВЕРСИТЕТ ИМЕНИ Н.Г. ЧЕРНЫШЕВСКОГО>> \\
\end{center}
\begin{center}
Кафедра системного анализа и автоматического управления
\end{center}

\begin{center}
Отчет по заданию 4. Вариант 10 и 29
\end{center}

\begin{flushleft}
Студента 3 курса 321 группы направления 09.03.01 ИВТ\\
Факультета компьютерных наук и информационных технологий\\
Чесакова Максима Евгеньевича
\end{flushleft}
%=================================================================================
\textbf{Задача №1}
Построить проекции в единичном квадрате плоскости \\ $\mathbb{I}^2 = [0,1] \times [0,1] \subset \mathbb{R}^2$ для функций $f$ и $h$ с точностью до $k = 15, \dots 20$ разрядов. Для функции $g$ записать обмотки тора и построить её график в $\mathbb{I}^2$. \\
$f(x) = 18 + x -7x^2$. \\
$
h(x) = (x \oplus  1) \oplus \\
 (2 (x \land (1 + 2x) \land (3+4x) \land (7+8x) \land (15+16x) \land (31+32x) \land (63+64x))) \oplus \\
 (4(x^2+ \textbf{29}))
$.
\\
$g(x) = \displaystyle\frac{17}{19}x - \frac{1}{15}$.

\textbf{Решение Часть 1.}
Графиком функции $f(x) = 18 + x -7x^2$ будет всюду плотное подмножество единичного квадрата (мера Лебега этого подмножества будет равна площади единичного квадрата). Иными словами, график функции $f$ суть вся поверхность тора $\mathbb{T}^2$. Проекции с точностью до $k=15,16,\dots,20$ разрядов на рисунках 1 --- 6.

\begin{figure}[!ht]
	\centering
	\includegraphics[width=13cm]{f(x)plot_k15.png}
\end{figure}
\begin{figure}[!ht]
	\centering
	\includegraphics[width=13cm]{f(x)plot_k16.png}
\end{figure}
\begin{figure}[!ht]
	\centering
	\includegraphics[width=13cm]{f(x)plot_k17.png}
\end{figure}
\begin{figure}[!ht]
	\centering
	\includegraphics[width=13cm]{f(x)plot_k18.png}
\end{figure}
\begin{figure}[!ht]
	\centering
	\includegraphics[width=13cm]{f(x)plot_k19.png}
\end{figure}
\begin{figure}[!ht]
	\centering
	\includegraphics[width=13cm]{f(x)plot_k20.png}
\end{figure}

График функции $h(x)$ также есть множество Лебеговой меры 1. Проекции с точностью до $k=15,16,\dots,20$ разрядов на рисунках 7 --- 12.

\begin{figure}[!ht]
	\centering
	\includegraphics[width=13cm]{plot_k15.png}
\end{figure}
\begin{figure}[!ht]
	\centering
	\includegraphics[width=13cm]{plot_k16.png}
\end{figure}
\begin{figure}[!ht]
	\centering
	\includegraphics[width=13cm]{plot_k17.png}
\end{figure}
\begin{figure}[!ht]
	\centering
	\includegraphics[width=13cm]{plot_k18.png}
\end{figure}
\begin{figure}[!ht]
	\centering
	\includegraphics[width=13cm]{plot_k19.png}
\end{figure}
\begin{figure}[!ht]
	\centering
	\includegraphics[width=13cm]{plot_k20.png}
\end{figure}

\textbf{Решение Часть 2.}

\textbf{Теорема.} \textit{
	График линейной p-адической функции $f(x) = cx + q$, где $c,q \in \mathbb{Z}_p \cap \mathbb{Q}$, $c=\dfrac{r}{s}$, $q=\dfrac{r'}{s'}$, $r,r'\in\mathbb{Z}$, $s,s'\in\mathbb{N}$, и при этом НОД(r,s) = НОД(r',s') = НОД(s, p) = НОД(s', p) = 1, представляет собой зацепление торических узлов и каждый торических узел является обмоткой тора (образом прямой единичного квадрата на поверхность тора):
	\[
	z_i = \{ (y \mod 1, (cy+e) \mod 1) : y \in \mathbb{R}, e \in C(q) \},
	\]
}
	где $i = 0,1,2,\dots,\text{mult}_m p-1, m = \dfrac{s'}{\text{НОД}(s, s')}$,\\ $ C(q)=\left\{
\bigl(-p^{\ell}\cdot \tfrac{r'}{s'}\bigr)\bmod 1
:\;
\ell=0,1,2,\ldots,\mathrm{mult}_m p - 1
\right\}.$

Количество обмоток тора равно $\mathrm{mult}_m p$,  где $m = \dfrac{s'}{\text{НОД}(s, s')}$.

Для функции \[
g(x) = \displaystyle\frac{17}{19}x - \frac{1}{15}
\]
$m = \dfrac{15}{\text{НОД}(15, 19)} = 15$, $\mathrm{mult}_{15} 2 = 4$.

\[
z_0  = \{ (y \bmod 1, (\frac{17}{19}y + \frac{1}{15}) \bmod 1) : y \in \mathbb{R} \}
\]
\[
z_0  = \{ (y \bmod 1, (\frac{17}{19}y + \frac{2}{15}) \bmod 1) : y \in \mathbb{R} \}
\]
\[
z_0  = \{ (y \bmod 1, (\frac{17}{19}y + \frac{4}{15}) \bmod 1) : y \in \mathbb{R} \}
\]
\[
z_0  = \{ (y \bmod 1, (\frac{17}{19}y + \frac{8}{15}) \bmod 1) : y \in \mathbb{R} \}
\]


\textbf{Программная реализация на Python.}

НОД вычисляем по алгоритму Евклида. На правом рисунке семейство параллельных прямых строится следующим образом: $i$-ая прямая получается из базового отрезк, на котором определяется $z_i$, сдвигом по оси $x$ на величину shift$ \in \{0,1,\dots,s-1\}$. На левом рисунке shift = 0. Для каждой обмотки изображены лишь те отрезки, которые целиком лежат в единичном квадрате.

Исходный код для $f$ и $h$ представлен в файле 4.py. Результаты его работы представлены на рисунках 1 --- 12.

Исходный код для $g$ представлен в файле 4-2.py. Результат его работы представлен на рисунке 13.

\begin{figure}[!ht]
	\centering
	\includegraphics[width=17cm]{g_from_belyaev_-1_15.png}
\end{figure}

\begin{Verbatim}[numbers=left]

\end{Verbatim}

%  \bibitem{Ione} GraphOnLine : [интерактивный онлайн-конструктор графов]. – URL: https://graphonline.ru/?graph=WnqNelghOjfEvOqsZZcst (дата обращения: 20.09.2025).
%  \label{сайт1}
%  \bibitem{Itwo} Источник 2
%\end{thebibliography}

%===============================================================================
\end{document}