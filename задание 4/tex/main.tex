\documentclass[coursework, och]{SCWorks1}
\usepackage[T2A]{fontenc}
\usepackage[utf8]{inputenc}
\usepackage{graphicx}

\usepackage[sort,compress]{cite}
\usepackage{amsmath}
\usepackage{amssymb}
\usepackage{amsthm}
\usepackage{fancyvrb}
\usepackage{longtable}
\usepackage{array}
\usepackage[english,russian]{babel}

\usepackage[colorlinks=false]{hyperref}

\newcommand{\eqdef}{\stackrel {\rm def}{=}}

\newtheorem{lem}{Лемма}
\newtheorem{definition}{Определение}

\begin{document}


\begin{center}
    МИНОБРНАУКИ РОССИИ\\
    Федеральное государственное бюджетное образовательное учреждение высшего образования \\
    <<САРАТОВСКИЙ НАЦИОНАЛЬНЫЙ ИССЛЕДОВАТЕЛЬСКИЙ ГОСУДАРСТВЕННЫЙ УНИВЕРСИТЕТ ИМЕНИ Н.Г. ЧЕРНЫШЕВСКОГО>> \\
\end{center}
\begin{center}
Кафедра системного анализа и автоматического управления
\end{center}

\begin{center}
Отчет по заданию 4. Вариант 10 и 29
\end{center}

\begin{flushleft}
Студента 3 курса 321 группы направления 09.03.01 ИВТ\\
Факультета компьютерных наук и информационных технологий\\
Чесакова Максима Евгеньевича
\end{flushleft}
%=================================================================================
\textbf{Задача №1}
Построить проекции в единичном квадрате плоскости \\ $\mathbb{I}^2 = [0,1] \times [0,1] \subset \mathbb{R}^2$ для функций $f$ и $h$ с точностью до $k = 15, \dots 20$ разрядов. Для функции $g$ записать обмотки тора и построить её график в $\mathbb{I}^2$. \\
$f(x) = 18 + x -7x^2$. \\
$
h(x) = (x \oplus  1) \oplus \\
 (2 (x \land (1 + 2x) \land (3+4x) \land (7+8x) \land (15+16x) \land (31+32x) \land (63+64x))) \oplus \\
 (4(x^2+ \textbf{29}))
$.
\\
$g(x) = \displaystyle\frac{17}{19}x - \frac{1}{15}$.

\textbf{Решение Часть 1.}


\begin{figure}[!ht]
	\centering
	\includegraphics[width=13cm]{f(x)plot_k15.png}
\end{figure}
\begin{figure}[!ht]
	\centering
	\includegraphics[width=13cm]{f(x)plot_k16.png}
\end{figure}
\begin{figure}[!ht]
	\centering
	\includegraphics[width=13cm]{f(x)plot_k17.png}
\end{figure}
\begin{figure}[!ht]
	\centering
	\includegraphics[width=13cm]{f(x)plot_k18.png}
\end{figure}
\begin{figure}[!ht]
	\centering
	\includegraphics[width=13cm]{f(x)plot_k19.png}
\end{figure}
\begin{figure}[!ht]
	\centering
	\includegraphics[width=13cm]{f(x)plot_k20.png}
\end{figure}



\begin{figure}[!ht]
	\centering
	\includegraphics[width=13cm]{plot_k15.png}
\end{figure}
\begin{figure}[!ht]
	\centering
	\includegraphics[width=13cm]{plot_k16.png}
\end{figure}
\begin{figure}[!ht]
	\centering
	\includegraphics[width=13cm]{plot_k17.png}
\end{figure}
\begin{figure}[!ht]
	\centering
	\includegraphics[width=13cm]{plot_k18.png}
\end{figure}
\begin{figure}[!ht]
	\centering
	\includegraphics[width=13cm]{plot_k19.png}
\end{figure}
\begin{figure}[!ht]
	\centering
	\includegraphics[width=13cm]{plot_k20.png}
\end{figure}

\textbf{Решение Часть 2.}




\textbf{Ответ:} .

\textbf{Программная реализация на Python.}

Исходный код представлен в файле 4.py. Результаты его работы представлены на рисунках 1.


\begin{Verbatim}[numbers=left]


\end{Verbatim}


%  \bibitem{Ione} GraphOnLine : [интерактивный онлайн-конструктор графов]. – URL: https://graphonline.ru/?graph=WnqNelghOjfEvOqsZZcst (дата обращения: 20.09.2025).
%  \label{сайт1}
%  \bibitem{Itwo} Источник 2
%\end{thebibliography}

%===============================================================================
\end{document}