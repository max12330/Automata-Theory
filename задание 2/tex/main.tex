\documentclass[coursework, och]{SCWorks1}
\usepackage[T2A]{fontenc}
\usepackage[utf8]{inputenc}
\usepackage{graphicx}

\usepackage[sort,compress]{cite}
\usepackage{amsmath}
\usepackage{amssymb}
\usepackage{amsthm}
\usepackage{fancyvrb}
\usepackage{longtable}
\usepackage{array}
\usepackage[english,russian]{babel}

\usepackage[colorlinks=false]{hyperref}

\newcommand{\eqdef}{\stackrel {\rm def}{=}}

\newtheorem{lem}{Лемма}
\newtheorem{definition}{Определение}

\begin{document}


\begin{center}
    МИНОБРНАУКИ РОССИИ\\
    Федеральное государственное бюджетное образовательное учреждение высшего образования \\
    <<САРАТОВСКИЙ НАЦИОНАЛЬНЫЙ ИССЛЕДОВАТЕЛЬСКИЙ ГОСУДАРСТВЕННЫЙ УНИВЕРСИТЕТ ИМЕНИ Н.Г. ЧЕРНЫШЕВСКОГО>> \\
\end{center}
\begin{center}
Кафедра системного анализа и автоматического управления
\end{center}

\begin{center}
Отчет по заданию 2. Вариант 10($29~\text{mod}~20+1=10$)
\end{center}

\begin{flushleft}
Студента 3 курса 321 группы направления 09.03.01 ИВТ\\
Факультета компьютерных наук и информационных технологий\\
Чесакова Максима Евгеньевича
\end{flushleft}
%=================================================================================
\textbf{Задача №1}
Проверить на биективность/транзитивность полиномы $f(x) = 18 + x -7x^2$ и $g(x) = \displaystyle\frac{17}{19}x - \frac{1}{15}$ на $\mathbb{Z}_2$. Решить задачу аналитически, затем проверить с помощью программной реализации.

\textbf{Аналитическое решение.}

\textbf{Теорема Ларина.} \textit{
    Многочлен $F$ с целыми или рациональными 2-адическими коэффициентами биективен (транзитивен) на $\mathbb{Z}_2$, тогда и только тогда, когда $F$ биективен (транзитивен) по модулю 4, т.~е. редукция $F \mod 4$ является перестановкой (транзитивен по модулю 8, т.~е. редукция $F \mod 8$ одноцикловая перестановка)
}

Найдём редукцию $f \mod 4$:\\ 
$18 = (0)^\infty10010$, ~~~~ $1 = (0)^\infty1$, ~~~~ $-7 = (1)^\infty001$, ~~ $\Rightarrow ~~ f \mod 4 = 2 + x + x^2$. \\
$f(0) \mod 4 = 2, ~~ f(1) \mod 4 = 0, ~~ f(2) \mod 4 = 0, ~~ f(3) \mod 4 = 2$.
\begin{figure}[!ht]
	\centering
	\includegraphics[width=5cm]{Диаграмма 1.png}
	\caption{диаграмма $f \mod 4$}
\end{figure} \\
Следовательно, $f$ не биективна на $\mathbb{Z}_2$. Следовательно не транзитивна на $\mathbb{Z}_2$.

Найдём редукцию $g \mod 4$:\\ 
Для разложения дробей в 2-адические числа используем лемму Малера:\\
\textbf{Лемма Малера.} \textit{
Пусть $r,s,p \in \mathbb{Z};~s \geq 1$ и пусть $p \geq 2$, НОД$(p,s) = 1$. Тогда существуют уникальные $A,R \in \mathbb{Z}$ такие, что $\displaystyle\frac{r}{s} = A + p \frac{R}{s}$, где $0 \leq A \leq p-1$.
}
С~помощью леммы имеем:\\
$--------------------$ \\
$\displaystyle\frac{17}{19} = 1 + 2 \cdot \frac{-1}{19}$, \\
$\displaystyle\frac{-1}{19} = 1 + 2 \cdot \frac{-10}{19}$, \\
$\displaystyle\frac{-10}{19} = 0 + 2 \cdot \frac{-5}{19}$, \\
$\displaystyle\frac{-5}{19} = 1 + 2 \cdot \frac{-12}{19}$, \\
$\displaystyle\frac{-12}{19} = 0 + 2 \cdot \frac{-6}{19}$, \\
$\displaystyle\frac{-6}{19} = 0 + 2 \cdot \frac{-3}{19}$, \\
$\displaystyle\frac{-3}{19} = 1 + 2 \cdot \frac{-11}{19}$, \\
$\displaystyle\frac{-11}{19} = 1 + 2 \cdot \frac{-15}{19}$, \\
$\displaystyle\frac{-15}{19} = 1 + 2 \cdot \frac{-17}{19}$, ~~~~~~~~~~ $\Longrightarrow ~~~ \displaystyle\frac{17}{19} = (0000110101111001011)^\infty$ \\
$\displaystyle\frac{-17}{19} = 1 + 2 \cdot \frac{-18}{19}$, \\
$\displaystyle\frac{-18}{19} = 0 + 2 \cdot \frac{-9}{19}$.\\
$\displaystyle\frac{-9}{19} = 1 + 2 \cdot \frac{-14}{19}$, \\
$\displaystyle\frac{-14}{19} = 0 + 2 \cdot \frac{-7}{19}$, \\
$\displaystyle\frac{-7}{19} = 1 + 2 \cdot \frac{-13}{19}$, \\
$\displaystyle\frac{-13}{19} = 1 + 2 \cdot \frac{-16}{19}$, \\
$\displaystyle\frac{-16}{19} = 0 + 2 \cdot \frac{-8}{19}$.\\
$\displaystyle\frac{-8}{19} = 0 + 2 \cdot \frac{-4}{19}$, \\
$\displaystyle\frac{-4}{19} = 0 + 2 \cdot \frac{-2}{19}$.\\
$\displaystyle\frac{-2}{19} = 0 + 2 \cdot \frac{-1}{19}$.\\
$--------------------$ \\
$\displaystyle\frac{-1}{15} = 1 + 2 \cdot \frac{-8}{15}$, \\
$\displaystyle\frac{-8}{15} = 0 + 2 \cdot \frac{-4}{15}$, \\
$\displaystyle\frac{-4}{15} = 0 + 2 \cdot \frac{-2}{15}$, ~~~~~~~~~~ $\Longrightarrow ~~~ \displaystyle\frac{-1}{15} = (0001)^\infty$ \\
$\displaystyle\frac{-2}{15} = 0 + 2 \cdot \frac{-1}{15}$, \\
$--------------------$ \\
$g \mod 4 = 3x + 1$.\\
$g(0) \mod 4 = 1, ~~ g(1) \mod 4 = 0, ~~ g(2) \mod 4 = 3, ~~ g(3) \mod 4 = 2$. \\
\begin{figure}[!ht]
	\centering
	\includegraphics[width=5cm]{Диаграмма 2.png}
	\caption{диаграмма $g \mod 4$}
\end{figure} \\
Следовательно, $g$ биективна на $\mathbb{Z}_2$. Проверим транзитивность. Найдём редукцию $g \mod 8$:\\
$g \mod 8 = 3x + 1$.\\
$g(0) \mod 8 = 1, ~~ g(1) \mod 8 = 4, ~~ g(2) \mod 8 = 7, ~~ g(3) \mod 8 = 2, \\ g(4) \mod 8 = 5, ~~ g(5) \mod 8 = 0, ~~ g(6) \mod 8 = 3, ~~ g(7) \mod 8 = 6$. \\
\begin{figure}[!ht]
	\centering
	\includegraphics[width=5cm]{Диаграмма 3.png}
	\caption{диаграмма $g \mod 8$}
\end{figure} \\
Функция $g \mod 8$ имеет двухцикловую перестановку, следовательно $g$ не транзитивна на $\mathbb{Z}_2$.

\textbf{Ответ:} $f$ не биективна, не транзитивна на $\mathbb{Z}_2$, $g$ биективна на $\mathbb{Z}_2$, но не транзитивна на $\mathbb{Z}_2$.

\textbf{Программная реализация на Python.}

\begin{Verbatim}[numbers=left]


\end{Verbatim}


%  \bibitem{Ione} GraphOnLine : [интерактивный онлайн-конструктор графов]. – URL: https://graphonline.ru/?graph=WnqNelghOjfEvOqsZZcst (дата обращения: 20.09.2025).
%  \label{сайт1}
%  \bibitem{Itwo} Источник 2
%\end{thebibliography}

%===============================================================================
\end{document}