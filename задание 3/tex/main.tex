\documentclass[coursework, och]{SCWorks1}
\usepackage[T2A]{fontenc}
\usepackage[utf8]{inputenc}
\usepackage{graphicx}

\usepackage[sort,compress]{cite}
\usepackage{amsmath}
\usepackage{amssymb}
\usepackage{amsthm}
\usepackage{fancyvrb}
\usepackage{longtable}
\usepackage{array}
\usepackage[english,russian]{babel}

\usepackage[colorlinks=false]{hyperref}

\newcommand{\eqdef}{\stackrel {\rm def}{=}}

\newtheorem{lem}{Лемма}
\newtheorem{definition}{Определение}

\begin{document}


\begin{center}
    МИНОБРНАУКИ РОССИИ\\
    Федеральное государственное бюджетное образовательное учреждение высшего образования \\
    <<САРАТОВСКИЙ НАЦИОНАЛЬНЫЙ ИССЛЕДОВАТЕЛЬСКИЙ ГОСУДАРСТВЕННЫЙ УНИВЕРСИТЕТ ИМЕНИ Н.Г. ЧЕРНЫШЕВСКОГО>> \\
\end{center}
\begin{center}
Кафедра системного анализа и автоматического управления
\end{center}

\begin{center}
Отчет по заданию 3. Вариант 29
\end{center}

\begin{flushleft}
Студента 3 курса 321 группы направления 09.03.01 ИВТ\\
Факультета компьютерных наук и информационных технологий\\
Чесакова Максима Евгеньевича
\end{flushleft}
%=================================================================================
\textbf{Задача №1}
Транзитивна ли заданная функция $h: \mathbb{Z}_2 \rightarrow \mathbb{Z}_2$? Решить задачу аналитически, затем проверить с помощью программной реализации транзитивность по модулю 256.

$
h(x) = (x \oplus  1) \oplus \\
 (2 (x \land (1 + 2x) \land (3+4x) \land (7+8x) \land (15+16x) \land (31+32x) \land (63+64x))) \oplus \\
 (4(x^2+ \textbf{29}))
$

\textbf{Аналитическое решение.}
$h(x)$ не является многочленом над $\mathbb{Z}_2$, следовательно, критерий транзиивности многочленов не работает. 

Функция $h$ есть композиция арифметических и логических операций: сложения, умножения, XOR и AND, следовательно, функция $h$ 1-Липшицева, поскольку композиция 1-Липшицевых функций --- 1-Липшицева.
Значит, нашу функцию $h$ можно записать в виде ряда 
\[
    h(x) = \sum_{i=0}^{\infty} \psi_i (x_0,...,x_i)2^i,
\]
где $\psi_i : \{0,1\}^{i+1} \rightarrow  \{0,1\}$.

Представим нашу функцию $h$ в виде поразрядной суммы трех функций
\[
    h(x) = F_0(x) \oplus 2F_1(x)\oplus 4F_2(x),
\]
где $F_0(x) = x \oplus  1$, \\
$F_1(x) = x \land (1 + 2x) \land (3+4x) \land (7+8x) \land (15+16x) \land (31+32x) \land (63+64x)$, \\
$F_2(x) = x^2+ \textbf{29}$.

Координатные функции для $F_0(x)$:
\[
\delta_0(F_0(x)) = \delta_0 (x) \oplus 1 = x_0 \oplus 1,
\]
\[
\delta_1(F_0(x)) = \delta_1 (x) = x_1,
\]
\[
\delta_2(F_0(x)) = \delta_2 (x) = x_2,
\]
\[
\dots
\]
\[
\delta_i(F_0(x)) = \begin{cases}
    x_0 \oplus 1, & \text{если } i=0 \\
    x_i, & \text{если } i>0.
\end{cases}
\]

Координатные функции для $2 F_1(x)$:
\[
\delta_0(2F_1(x)) = 0,\qquad 
\delta_i(2F_1(x)) = \delta_{i-1}(F_1(x)), 
\quad \text{при } i = 1,2,\ldots
\]
\[
\delta_0(F_0 \oplus (2F_1(x))) = \delta_0(x) \oplus 1,
\]
\[
\delta_i(F_0 \oplus (2F_1(x))) 
= \delta_i(x) \oplus \delta_{i-1}(F_1(x)), 
\quad \text{для } i = 1,2,\ldots
\]

Координатные функции для $F_1(x)$:
\begin{align*}
\delta_0(F_1(x)) &= \delta_0(x); \\
\delta_1(F_1(x)) &= \delta_1(x)\cdot \delta_0(x); \\
\delta_2(F_1(x)) &= \delta_2(x)\cdot \delta_1(x)\cdot \delta_0(x); \\
\delta_3(F_1(x)) &= \delta_3(x)\cdot \delta_2(x)\cdot \delta_1(x)\cdot \delta_0(x); \\
&\ \dots \\
\delta_6(F_1(x)) &= \delta_6(x)\cdot \ldots \cdot \delta_1(x)\cdot \delta_0(x).
\end{align*}

В общем виде:
\[
\delta_i(F_1(x)) 
= \delta_i(x)\cdot \ldots \cdot \delta_{\max\{0,\,i-6\}}(x).
\]
Координатные функции для $4 F_2(x)$:
\[
\delta_0(4F_2(x)) = 0, \qquad \delta_1(4F_2(x)) = 0, \qquad \delta_i(4F_2(x)) = \delta_{i-2}(F_2(x)), \quad \text{при } i = 2,3,\ldots
\]

Координатные функции для $h$:
\[
\delta_0(h(x)) = \delta_0(x) \oplus 1 = 1 \oplus x_0;
\]
\[
\delta_1(h(x)) = \delta_1(x) \oplus \delta_0(x) = x_1 \oplus x_0;
\]
\[
\delta_i(h(x)) = \delta_i(x) \oplus [\delta_{i-1}(x)\cdot \ldots \cdot \delta_{\max\{0,\,i-7\}}(x)] \oplus \delta_{i-2}(F_2(x)), i=2,3,\dots
\]

Заметим, что $\delta_{i-2}(F_2(x))$ не зависит от $\delta_{i-1}(x)$.

В итоге имеем:
\begin{equation}
h(x) = \sum_{i=0}^{\infty} \psi_i (x_0,...,x_i)2^i,    
\end{equation}

Здесь
\[
\psi_0(x_0) = 1 \oplus \delta_0(x) = 1 \oplus x_0,
\]
\[
\psi_1(x_0, x_1) = \delta_1(x) \oplus \delta_0(x) = x_1 \oplus x_0,
\]
\[
\psi_i(x_0, \ldots, x_i) 
= \delta_i(f(x)) 
= x_i \oplus \Phi_i(x_0, \ldots, x_{i-1}),
\]
где $\Phi_i(x_0, \ldots, x_{i-1})$ есть АНФ степени $i$ и 
$\Phi_0(x_0) = 1$.

Представление функций из $\mathbb{Z}_2$ в $\mathbb{Z}_2$ 
с помощью координатных функций позволяет определить, 
принадлежит ли функция классу сохраняющих меру 
или эргодических функций. При этом используется 
установленный фольклорный критерий для 
2-адических 1-липшицевых функций.

\textbf{Теорема 1 (фольклор).}
Функция $f$, определенная равенством (1), сохраняет меру тогда и только тогда,
когда для каждого $i = 0,1,2,\ldots$ АНФ $i$-й координатной функции есть
\[
\psi_i(x_0,\ldots,x_i)
= x_i \oplus \phi_i(x_0,\ldots,x_{i-1}),
\]
где $\phi_i$ есть АНФ от булевой функции от булевых переменных
$x_0,\ldots,x_{i-1}$ и $\phi_0$ есть константа из $\{0,1\}$.

Функция $f$ эргодична тогда и только тогда, когда, дополнительно,
$\phi_0 = 1$, и каждая булева функция $\phi_i$ имеет нечетный вес,
т.е.\ принимает значение $1$ в точности в нечетном числе точек
из $\{0,1\}^i$ для $i = 1,2,\ldots$.
Последнее условие выполнено тогда и только тогда,
когда степень АНФ $\phi_i$ для $i \ge 1$ в точности равна $1$,
т.е.\ тогда и только тогда, когда АНФ $\phi_i$
содержит моном $x_0 \cdots x_{i-1}$.

\textbf{Теорема 2 (основная эргодическая).}
Пусть функция $f: \mathbb{Z}_p^n \to \mathbb{Z}_p^m$, $m \le n$, есть $1$-липшицева функция. Если $m = n$, функция $f$ сохраняет меру $\mu_p$ (является эргодической) тогда и только тогда, когда $f$ биективна по модулю $p^k$ (транзитивна по модулю $p^k$) при всех $k = 1, 2, 3, \ldots$ (при достаточно больших значениях $k$). При $m \le n$ функция $f$ сохраняет меру $\mu_p$ тогда и только тогда, когда $f$ сбалансирована по модулю $p^k$ при всех $k = 1, 2, 3, \ldots$ (при достаточно больших значениях $k$).


Из построения по теореме 1 следует, что исследуемая 1-Липшицева функция $h(x)$ эргодична, из чего по теореме 2 следует её транзитивность в целом и по модулю 256 в частности.


\textbf{Ответ:} Да. функция транзитивна.

\textbf{Программная реализация на Python.}
Полный код программы приведён в файле.

Программа вычисляет вычеты по модулю 256 и проверет, составляют ли они одноцикловую перестановку. Результат работы программы представлен на рисунке.

\begin{figure}[!ht]
	\centering
	\includegraphics[width=15cm]{3.png}
	\caption{Проверка $h(x)$}
\end{figure}

\begin{Verbatim}[numbers=left]

\end{Verbatim}

%  \bibitem{Ione} GraphOnLine : [интерактивный онлайн-конструктор графов]. – URL: https://graphonline.ru/?graph=WnqNelghOjfEvOqsZZcst (дата обращения: 20.09.2025).
%  \label{сайт1}
%  \bibitem{Itwo} Источник 2
%\end{thebibliography}

%===============================================================================
\end{document}