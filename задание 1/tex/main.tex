\documentclass[coursework, och, times]{SCWorks1}
\usepackage[T2A]{fontenc}
\usepackage[utf8]{inputenc}
\usepackage{graphicx}

\usepackage[sort,compress]{cite}
\usepackage{amsmath}
\usepackage{amssymb}
\usepackage{amsthm}
\usepackage{fancyvrb}
\usepackage{longtable}
\usepackage{array}
\usepackage[english,russian]{babel}

\usepackage[colorlinks=false]{hyperref}

\newcommand{\eqdef}{\stackrel {\rm def}{=}}

\newtheorem{lem}{Лемма}
\newtheorem{definition}{Определение}

\begin{document}


\begin{center}
    МИНОБРНАУКИ РОССИИ
\end{center}
\begin{center}
    Федеральное государственное бюджетное образовательное учреждение высшего образования
\end{center}
\begin{center}
 <<САРАТОВСКИЙ НАЦИОНАЛЬНЫЙ ИССЛЕДОВАТЕЛЬСКИЙ ГОСУДАРСТВЕННЫЙ УНИВЕРСИТЕТ ИМЕНИ Н.Г. ЧЕРНЫШЕВСКОГО>>
 \end{center}
\begin{center}
Кафедра системного анализа и автоматического управления
\end{center}
\begin{center}
Отчет по заданию 1. Вариант 14($29~\text{mod}~16+1=14$)
\end{center}

\begin{flushleft}
Студента 3 курса 321 группы
направления 09.03.01 ИВТ

Факультета компьютерных наук и информационных технологий

Чесакова Максима Евгеньевича
\end{flushleft}
%=================================================================================
\textbf{Задача №1}
Построить фрагмент двоичного дерева и диаграмму Мура для 2-адической линейной функции $f(x) = 1 - 3x$.

Решение:
\begin{enumerate}
    \item 
Заметим, что функция вида $f(x) = ax + b$, где $a, b \in \mathbb{Z}_2$ является 1-Липшицевой, т.е. удовлетворяет условию Липшица с константой 1:
\[
|f(x) - f(y)|_2 = |ax + b - (ay + b)|_2 = |ax - ay|_2 = |a(x - y)_2| = |a|_2 \cdot |x - y|_2 \leq |x - y|_2.
\]
1-Липшицева функция является автоматной функцией, иными словами, она задает некоторый автомат. Если коэффициенты $a$ и $b$ лежат в $\mathbb{Z}_2\cap \mathbb{Q}$, т.е. $a$ и $b$ --- рациональные целые 2-адические числа, то функция $f(x) = ax + b$ задает КОНЕЧНЫЙ автомат. Для нашей функции $a=-3$, $b=1$, значит функция $f(x)=1 - 3x$ задает конечный автомат, граф переходов которого и требуется найти.
\item 
В таблице~\ref{table1} выпишем значения функции $f(x) = 1 - 3x$:
    \item 
Построим дерево переходов [\ref{сайт1}]:
\begin{figure}[!ht]
	\centering
	\includegraphics[width=19cm]{Дерево.png}
	\caption{дерево переходов}
\end{figure}

    \item 
Преобразуем в диаграмму Мура:

Очевидно, что s2 = s4, s3 = s6, s11 = s23, s22 = s5.
\begin{figure}[!ht]
	\centering
	\includegraphics[width=10cm]{Диаграмма Мура.png}
	\caption{диаграмма Мура}
\end{figure}

\begin{table}[!ht]
    \caption{Значения $f(x)$}
    \label{table1}
\begin{tabular}{r r l l r r l l}
  \hline
  $x_{10}$ & $x_2$ & $f_2(x)$ & $f_{10}(x)$ & $x_{10}$ & $x_2$ & $f_2(x)$ & $f_{10}(x)$ \\
  \hline
-16 &1111110000& 0000110001& 49 & 5 & 0000000101 & 1111110010 & -14 \\
-15	&1111110001& 0000101110& 46 & 6 & 0000000110 & 1111101111 & -17 \\
-14	&1111110010& 0000101011& 43 & 7 & 0000000111 & 1111101100 & -20 \\
-13 &1111110011& 0000101000& 40 & 8 & 0000001000 & 1111101001 & -23 \\
-12 &1111110100& 0000100101& 37 & 9 & 0000001001 & 1111100110 & -26 \\
-11	&1111110101& 0000100010& 34 & 10 & 0000001010 & 1111100011 & -29 \\
-10 &1111110110& 0000011111& 31 & 11 & 0000001011 & 1111100000 & -32 \\
-9	&1111110111& 0000011100& 28 & 12 & 0000001100 & 1111011101 & -35 \\
-8 &1111111000&	0000011001&	25 & 13 & 0000001101 & 1111011010 & -38 \\
-7 &1111111001&	0000010110&	22 & 14 & 0000001110 & 1111010111 & -41 \\
-6 &1111111010&	0000010011&	19 & 15 & 0000001111 & 1111010100 & -44 \\
-5 &1111111011&	0000010000&	16 & 16 & 0000010000 & 1111010001 & -47 \\
-4 &1111111100&	0000001101&	13 & 17 & 0000010001 & 1111001110 & -50 \\
-3 &1111111101&	0000001010&	10 & 18 & 0000010010 & 1111001011 & -53 \\
-2 &1111111110&	0000000111&	7 & 19 & 0000010011 & 1111001000 & -56 \\
-1 &1111111111&	0000000100&	4 & 20 & 0000010100 & 1111000101 & -59 \\
0 &	0000000000&	0000000001&	1 & 21 & 0000010101 & 1111000010 & -62 \\
1 &	0000000001&	1111111110&	-2 & 22 & 0000010110 & 1110111111 & -65 \\
2 &	0000000010&	1111111011&	-5 & 23 & 0000010111 & 1110111100 & -68 \\
3 &	0000000011&	1111111000&	-8 & 24 & 0000011000 & 1110111001 & -71 \\
4 &	0000000100&	1111110101&	-11 & -- & --        &  --        & -- \\
  \hline
\end{tabular}

\end{table}



\end{enumerate}

~
\begin{thebibliography}{99}
  \bibitem{Ione} GraphOnLine : [интерактивный онлайн-конструктор графов]. – URL: https://graphonline.ru/?graph=WnqNelghOjfEvOqsZZcst (дата обращения: 20.09.2025).
  \label{сайт1}
  %\bibitem{Itwo} Источник 2
\end{thebibliography}

%===============================================================================
\end{document}